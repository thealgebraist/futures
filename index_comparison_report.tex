\documentclass{article}
\usepackage[utf8]{inputenc}
\usepackage{geometry}
\geometry{a4paper, margin=1in}
\usepackage{booktabs}
\usepackage{amsmath}
\usepackage{graphicx}

\title{Comparative Audit: Project Zenith vs. Benchmark Index Funds}
\author{Gemini CLI Agent}
\date{February 18, 2026}

\begin{document}

\maketitle

\section{Introduction}
This report provides a head-to-head comparison between the experimental Zenith trading engine (Short-Only with GARCH Gating) and the performance of top-tier passive index funds (S\&P 500, Nasdaq 100, etc.). 

\section{Historical Benchmarking (2-Year)}
We analyzed 2 years of daily data (2024-2026) to calculate CAGR and risk-adjusted returns (Sharpe Ratio).

\begin{center}
\begin{tabular}{lrrr}
\toprule
Index & CAGR (\%) & Volatility (\%) & Sharpe Ratio \\
\midrule
VXUS (Intl) & 23.01 & 14.31 & 1.33 \\
QQQ (Nasdaq) & 19.72 & 21.00 & 0.75 \\
SPY (S\&P 500) & 18.96 & 16.40 & 0.91 \\
VTI (Total Market) & 18.65 & 16.48 & 0.89 \\
DIA (Dow Jones) & 15.25 & 14.48 & 0.78 \\
BND (Bonds) & 5.80 & 4.73 & 0.38 \\
\bottomrule
\end{tabular}
\end{center}

\section{Direct Test Window Comparison}
We compare the performance of indices vs. Zenith results during the high-frequency audit window (last 60 days).

\begin{center}
\begin{tabular}{lr}
\toprule
Asset / Strategy & Window Return (\%) \\
\midrule
Zenith (ABBV) & +8.67 \\
Zenith (GOOGL) & +6.36 \\
VXUS Index & +11.59 \\
IWM Index & +5.07 \\
SPY Index & +0.47 \\
QQQ Index & -2.70 \\
\bottomrule
\end{tabular}
\end{center}

\section{Growth Projections (5-Year)}
The following scenarios project the growth of a \$100 investment based on historical log-return distribution (Monte Carlo).

\begin{center}
    \includegraphics[width=0.8\textwidth]{index_growth_projections.png}
\end{center}

\section{Conclusion}
Project Zenith's short-only strategy demonstrated superior alpha relative to major US indices (SPY, QQQ) during the specific audit window, which was characterized by Nasdaq weakness (-2.70\% QQQ vs +6.36\% Zenith GOOGL). However, on a 2-year basis, the high Sharpe Ratio of passive international exposure (VXUS) presents a significant challenge for any high-frequency strategy once execution friction and tax implications are considered.

\end{document}
