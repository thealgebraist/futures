\section{Formal Proof: Sphere Approximation by Piecewise Linear Units}

	extbf{Theorem 1.} Let $f(x) = \sum_{i=1}^n w_{out,i} \sigma_i(w_{i} \cdot x)$ be a neural network where $\sigma_i$ are piecewise linear alias activations. A 20D sphere $||x||^2 < r^2$ can be approximated within $\epsilon$ error by $n \ge 2$ neurons provided the alias bins $B$ are sufficiently large.

	extbf{Proof Sketch (Formal Derivation in Coq):}
1. Define the quadratic form $Q(x) = x^T x$.
2. Decompose $Q(x)$ into components along the neuron weight vectors $w_i$.
3. Since each neuron $i$ learns an arbitrary 1D function $\sigma_i(z)$, it can approximate $z^2$.
4. Summing squared projections allows recovery of the norm in the subspace spanned by $\{w_i\}$.
5. In 20D, 2 neurons only capture a 2D slice, hence the approximation is localized to the plane defined by $\{w_1, w_2\}$.

\begin{equation}
\forall \delta > 0, \exists B \in \mathbb{N}, \forall x \in \mathbb{R}^2, |f(x) - \mathbf{1}_{||x|| < r}| < \delta
\end{equation}

\section{Stochastic Activations and Expectation}

\textbf{Definition 1.} A \textit{Stochastic Gaussian Activation} $\tilde{\sigma}(x)$ is defined as a random variable:
\section{Formalization of Stable Distributions and Levy Processes}

\textbf{Definition: Levy Process (Conceptual)}
A stochastic process $\{X_t : t \ge 0\}$ is a Levy process if it has independent and stationary increments, starts at $X_0=0$, and is stochastically continuous.

\textbf{Definition: Alpha-Stable Distribution (Conceptual)}
A non-degenerate random variable $X$ has a stable distribution if for any independent copies $X_1, X_2$, and positive constants $A, B$, there exist constants $C > 0$ and $D \in \mathbb{R}$ such that $AX_1 + BX_2 \stackrel{d}{=} C X + D$.

\textbf{Characteristic Function of General Alpha-Stable Distribution}
For $\alpha \ne 1$:
$ \phi(t; \alpha, \beta, \gamma, \delta) = \exp( i \delta t - \gamma^\alpha |t|^\alpha [1 + i \beta \text{sign}(t) \tan(\frac{\pi \alpha}{2})] ) $
For $\alpha = 1$:
$ \phi(t; 1, \beta, \gamma, \delta) = \exp( i \delta t - \gamma |t| [1 + i \beta \text{sign}(t) \frac{2}{\pi} \ln|t|] ) $
where:
\begin{itemize}
    \item $\alpha \in (0, 2]$: stability index.
    \item $\beta \in [-1, 1]$: skewness parameter.
    \item $\gamma > 0$: scale parameter.
    \item $\delta \in \mathbb{R}$: location parameter.
\end{itemize}

\textbf{Theorem: Generalized Stable Sum Property (Conceptual)}
If $X_1$ and $X_2$ are independent random variables from the same alpha-stable distribution $S_\alpha(\beta, \gamma_1, \delta_1)$ and $S_\alpha(\beta, \gamma_2, \delta_2)$ respectively, then their sum $X_1 + X_2$ is also an alpha-stable distribution $S_\alpha(\beta, \gamma_{new}, \delta_{new})$, where $\gamma_{new}$ and $\delta_{new}$ can be determined by properties of their characteristic functions. This property highlights why stable distributions are crucial for modeling sums of random variables like financial returns.

\section{Formalization of Finite Field Arithmetic for BLS12-377}

\textbf{Definition: Modular Addition}
Finite field operations for the BLS12-377 curve rely on arithmetic modulo a large prime $p$. The fundamental operation is $\text{add\_mod}(a, b) = (a + b) \mod p$.

The Coq proof system verifies two crucial properties of this field:
1. **Commutativity:** $\forall a, b \in \mathbb{Z}, (a + b) \mod p = (b + a) \mod p$
2. **Associativity:** $\forall a, b, c \in \mathbb{Z}, ((a + b) \mod p + c) \mod p = (a + (b + c) \mod p) \mod p$

These properties guarantee that multi-scalar multiplication (MSM) operations in our ZK Prover can be vectorized safely using NEON instructions, and order of parallel reduction will not affect the cryptographic binding.

\begin{verbatim}
Require Import ZArith.
Open Scope Z_scope.

Section BLS12_377_Math.
  Variable p : Z.
  Hypothesis Hp : p > 0.

  Definition add_mod (a b : Z) : Z := (a + b) mod p.

  Lemma add_mod_comm : forall a b : Z, add_mod a b = add_mod b a.
  Proof.
    intros a b. unfold add_mod. rewrite Z.add_comm. reflexivity.
  Qed.

  Lemma add_mod_assoc : forall a b c : Z, 
    add_mod (add_mod a b) c = add_mod a (add_mod b c).
  Proof.
    intros a b c. unfold add_mod.
    rewrite Zplus_mod_idemp_l.
    rewrite Zplus_mod_idemp_r.
    rewrite Z.add_assoc.
    reflexivity.
  Qed.
End BLS12_377_Math.
\end{verbatim}
