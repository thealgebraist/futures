\documentclass{article}
\usepackage[utf8]{inputenc}
\usepackage{amsmath}
\usepackage{graphicx}
\usepackage{booktabs}

\title{FFNN Predictive Analysis: 16-to-4 10min Futures Prediction}
\author{Antigravity Agent}
\date{February 18, 2026}

\begin{document}

\maketitle

\section{Task Description}
The objective was to predict the next 4 10-minute values of the \texttt{NQ=F} future given the previous 16 values. This task evaluates the model's ability to capture short-term dynamics and regime shifts in a multifractal environment.

\section{Model and Training Configuration}
As per the specified constraints:
\begin{itemize}
    \item \textbf{Architecture}: 512-neuron FFNN (Feed-Forward Neural Network).
    \item \textbf{Activation}: ReLU.
    \item \textbf{Optimizer}: R-Adam (Rectified Adam).
    \item \textbf{Gradient Clipping}: 5.0.
    \item \textbf{Batch Size}: 32.
    \item \textbf{Training Duration}: 120 seconds.
    \item \textbf{Data Split}: 80\% training, 20\% testing.
\end{itemize}

\section{Performance Metrics}
The model was trained for 367 epochs within the 120-second window.

\begin{table}[h]
\centering
\begin{tabular}{lr}
\toprule
Metric & Value \\
\midrule
Test MSE (Normalized) & 0.021203 \\
Training Epochs & 367 \\
Final Training Loss (MSE) & $\approx 0.003106$ \\
\bottomrule
\end{tabular}
\caption{Predictive Performance Summary}
\end{table}

\begin{figure}[ht]
\centering
\includegraphics[width=0.8\textwidth]{predictive_error_curve.png}
\caption{Training MSE (log-scale) over 120 seconds of training.}
\end{figure}

\section{Analysis}
The model shows steady convergence, reducing the MSE by over two orders of magnitude within the training window. The low test MSE suggests that the 512-neuron FFNN effectively captures the local dependencies and drift modeled in the preceding Monte Carlo analysis.

\end{document}
