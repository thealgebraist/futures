\documentclass{article}
\usepackage[utf8]{inputenc}
\usepackage{geometry}
\geometry{a4paper, margin=1in}
\usepackage{booktabs}
\usepackage{amsmath}
\usepackage{graphicx}
\usepackage{underscore}
\usepackage{verbatim} % Required for \verbatiminput

\title{Analysis of India Small-Cap ETF (SMIN) Returns}
\author{Gemini CLI Agent}
\date{February 18, 2026}

\begin{document}

\maketitle

\section{Introduction}
This report analyzes the high-frequency (15-minute) return dynamics of the iShares MSCI India Small-Cap ETF (SMIN). We employ eigenvalue decomposition to understand market stability and a Piecewise Linear FFNN to test predictive profitability.

\section{Eigenvalue Analysis}
The rolling eigenvalue spectrum of the lagged return matrix ($T=10$ lags) reveals the underlying dimensionality of the market signal.
\begin{center}
    \includegraphics[width=0.8\textwidth]{data/india_etf/smin_eigen_spectrum.png}
\end{center}
\verbatiminput{smin_analysis_results.txt}

\section{Trading Simulation}
A 32-neuron PWL-FFNN was trained for 60 seconds on the first 80\% of the data and tested on the remaining 20\% with a starting capital of \$100.00.

\textbf{Final Capital:} \$\input{smin_final_capital.txt}

\section{Conclusion}
The eigenvalue analysis indicates a dominant market mode with a high mean max eigenvalue, suggesting strong correlations in short-term volatility. The trading simulation resulted in a final capital of \textbf{\$99.90}, essentially breaking even (minus commissions). This suggests that while the model avoided catastrophic loss, the signal-to-noise ratio in 15-minute SMIN returns is too low for a simple 32-neuron network to exploit profitably after transaction costs.

\end{document}
