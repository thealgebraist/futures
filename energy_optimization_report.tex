\documentclass{article}
\usepackage[utf8]{inputenc}
\usepackage{geometry}
\geometry{a4paper, margin=1in}
\usepackage{booktabs}
\usepackage{amsmath}

	itle{Energy Trading Optimization: 8-Iteration FFNN Analysis}
\author{Gemini CLI Agent}
\date{February 18, 2026}

\begin{document}

\maketitle

\section{Introduction}
This report details the results of an 8-iteration optimization process for indirect energy trading. We evaluated eight different assets (ETFs, Stocks, and Futures) using a 32-neuron Feed-Forward Neural Network (FFNN) implemented in C++23. The goal was to achieve a profit from a starting capital of \$100.00 over a simulated month of unseen data.

\section{Methodology}
\begin{itemize}
    \item 	extbf{Model:} 32-neuron FFNN with Tanh activation.
    \item 	extbf{Data:} 15-minute interval returns for 60 days.
    \item 	extbf{Training:} 60 seconds per iteration using R-Adam and Gradient Clipping.
    \item 	extbf{Optimization Loop:} After each iteration, the learning rate was adjusted based on the PnL of the previous asset to "improve" the training profile for the next.
\end{itemize}

\section{Results}
The following table summarizes the final capital after one month of simulated trading for each asset.

\begin{center}
\begin{tabular}{lrr}
	oprule
Asset & Final Capital (\$) & Final Learning Rate 
\midrule
USO (Oil ETF) & 75.67 & 0.001000 
UNG (Nat Gas ETF) & 87.76 & 0.000500 
XLE (Energy Sector) & 83.90 & 0.000250 
XOM (Exxon Mobil) & 68.79 & 0.000125 
CVX (Chevron) & 100.00 & 0.000063 
NEE (NextEra) & 79.89 & 0.000075 
CL=F (Crude Oil) & 98.61 & 0.000038 
NG=F (Nat Gas) & 89.56 & 0.000019 
\bottomrule
\end{tabular}
\end{center}

\section{Analysis}
\begin{enumerate}
    \item 	extbf{Profitability Challenge:} None of the iterations resulted in a significant net profit above the starting \$100.00. CVX maintained the capital, while others saw drawdowns of 10-30\%.
    \item 	extbf{Learning Rate Sensitivity:} Conservative learning rates (e.g., $1.8 	imes 10^{-5}$) showed slightly better stability in futures (CL=F, NG=F) compared to more aggressive rates used earlier in the loop.
    \item 	extbf{Asset Comparison:} Crude Oil futures (CL=F) and Chevron (CVX) were the most resilient to the FFNN's predictive noise, finishing closest to the initial balance.
\end{enumerate}

\section{Conclusion}
Indirect energy trading using a simple 32-neuron FFNN on 15-minute data is highly challenging for a \$100 account. The primary obstacle remains the high signal-to-noise ratio in intraday returns and the erosion of capital from commissions/slippage. More sophisticated architectures or longer-term timeframes may be required for consistent profitability.

\end{document}
