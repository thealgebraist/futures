\documentclass{article}
\usepackage[utf8]{inputenc}
\usepackage{amsmath}
\usepackage{graphicx}
\usepackage{booktabs}

\title{Trading Strategy Backtest: Monthly FFNN Simulation}
\author{Antigravity Agent}
\date{February 18, 2026}

\begin{document}

\maketitle

\section{Methodology}
The backtest utilized a 512-neuron FFNN model trained to predict the next 4 10-minute price windows. The simulation started with an initial capital of \$100 and executed a "long/short" strategy based on the model's predicted future average price relative to the current spot price.

\section{Formal Basis}
The portfolio accumulation logic was formalized in Coq, ensuring that $V_{t+1} = V_t(1 + r_t p_t)$ and verifying the non-negativity of the portfolio value under controlled leverage and returns.

\section{Simulation Results}
The simulation was conducted over a test set comprising 2074 steps (approximately 14.4 trading days).

\begin{table}[h]
\centering
\begin{tabular}{lr}
\toprule
Metric & Value \\
\midrule
Initial Capital & \$100.00 \\
Final Portfolio Value & \$99.01 \\
Total Return & -0.99\% \\
Number of Trades/Steps & 2074 \\
Trading Frequency & 10-minute intervals \\
\bottomrule
\end{tabular}
\caption{Backtest Performance Summary}
\end{table}

\begin{figure}[ht]
\centering
\includegraphics[width=0.8\textwidth]{equity_curve.png}
\caption{Equity curve and corresponding model positions (1:Long, -1:Short, 0:Cash).}
\end{figure}

\section{Discussion}
The strategy resulted in a marginal loss of \$0.99 over the simulation period. This drawdown corresponds to the baseline "no-edge" or slightly negative drift observed in the specific test window of the NQ=F future. Further refinement of the signal threshold or inclusion of transaction costs/slippage (not modeled here) would be necessary for a production-ready system.

\end{document}
