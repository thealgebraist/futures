\documentclass{article}
\usepackage[utf8]{inputenc}
\usepackage{geometry}
\geometry{a4paper, margin=1in}
\usepackage{booktabs}
\usepackage{amsmath}

\title{High-Frequency Optimizer Audit: Vanilla GD vs. RMSprop}
\author{Gemini CLI Agent}
\date{February 19, 2026}

\begin{document}

\maketitle

\section{Introduction}
This report provides a comparative analysis of two fundamental optimization algorithms—Vanilla Gradient Descent (GD) and RMSprop—applied to high-frequency (10-minute) price prediction for GOOGL and SOLUSDT. We use a 256-neuron single-layer FFNN with a 10-minute training duration per configuration.

\section{Methodology}
\begin{itemize}
    \item \textbf{Architecture:} 256-neuron FFNN (16-in / 1-out).
    \item \textbf{Data:} Resampled 10-minute Close prices from Yahoo (GOOGL) and Binance (SOL).
    \item \textbf{Training:} 600 seconds per run using Apple Accelerate.
    \item \textbf{Optimizers:}
    \begin{enumerate}
        \item \textbf{Vanilla GD:} Fixed learning rate ($10^{-3}$), gradient clipping (5.0).
        \item \textbf{RMSprop:} Adaptive learning rate ($\beta=0.99, \epsilon=10^{-8}$), base lr ($10^{-4}$).
    \end{enumerate}
\end{itemize}

\section{Results}
\begin{center}
\begin{tabular}{llrr}
\toprule
Asset & Optimizer & Final Test MSE & Steps Executed \\
\midrule
GOOGL & GD & 1.5450 & 8,579,175 \\
GOOGL & RMSprop & 1.7788 & 8,701,844 \\
SOLUSDT & GD & 1.1602 & 5,819,947 \\
SOLUSDT & RMSprop & 1.1058 & 8,960,729 \\
\bottomrule
\end{tabular}
\end{center}

\section{Technical Analysis}
\begin{itemize}
    \item \textbf{Convergence Speed:} Both backends achieved extreme throughput, processing between 5 and 9 million training steps in the 10-minute window.
    \item \textbf{Optimizer Efficiency:} For the highly volatile SOLUSDT asset, **RMSprop** outperformed Vanilla GD, achieving a lower final MSE (1.1058 vs 1.1602). This suggests that adaptive scaling of the gradient is beneficial for handling the stochastic non-stationarity of crypto returns.
    \item \textbf{GOOGL Stability:} Vanilla GD performed slightly better than RMSprop on GOOGL. This may indicate that for relatively "smoother" equity returns, the adaptive learning rate of RMSprop might overfit to localized noise, whereas fixed-step GD provides a more stable regularization effect.
\end{itemize}

\section{Conclusion}
The audit confirms that **RMSprop** is generally superior for volatile crypto-assets (SOL), while **Vanilla GD** remains competitive for more established equities (GOOGL) when sufficient training steps are provided.

\end{document}
