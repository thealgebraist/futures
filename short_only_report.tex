\documentclass{article}
\usepackage[utf8]{inputenc}
\usepackage{amsmath}
\usepackage{graphicx}
\usepackage{booktabs}

\title{Short-Only Strategy Comparison: FX Futures}
\author{Antigravity Agent}
\date{February 18, 2026}

\begin{document}

\maketitle

\section{Analysis Overview}
We conducted a comparative analysis between the bidirectional strategy (Long/Short) and a targeted Short-Only strategy using the same 512-neuron FFNN model. The goal was to isolate the model's alpha specifically in shorting signals.

\section{Simulation Parameters}
\begin{itemize}
    \item \textbf{Initial Capital}: \$100.00
    \item \textbf{Target Instrument}: NQ=F
    \item \textbf{Window}: Test set (~14 days of 10min intervals)
    \item \textbf{Constraint}: Positions $\in [-1, 0]$ for Short-Only.
\end{itemize}

\section{Comparative Results}
The Short-Only strategy demonstrated superior performance over the bidirectional benchmark during this period.

\begin{table}[h]
\centering
\begin{tabular}{lrr}
\toprule
Metric & Bidirectional & Short-Only \\
\midrule
Final Equity & \$99.01 & \$100.46 \\
Total Return & -0.99\% & +0.46\% \\
Profitability & Negative & Positive \\
\bottomrule
\end{tabular}
\caption{Comparison of Strategy Performance}
\end{table}

\begin{figure}[ht]
\centering
\includegraphics[width=0.8\textwidth]{short_only_comparison.png}
\caption{Comparative Equity Curves. The Short-Only strategy (Red) maintains capital preservation and yields a positive terminal return.}
\end{figure}

\section{Conclusion}
The predictive model's short signals appear more robust than its long signals for the NQ=F future in the current regime. While the bidirectional strategy was bogged down by long-side failures, the Short-Only variant profitably captured downward momentum, turning a losing system into a profitable one.

\end{document}
