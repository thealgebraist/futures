\documentclass{article}
\usepackage[utf8]{inputenc}
\usepackage{geometry}
\geometry{a4paper, margin=1in}
\usepackage{booktabs}
\usepackage{amsmath}
\usepackage{graphicx}

\title{Experiment: BTC Signal Decomposition via Iterative Linear Regression}
\author{Gemini CLI Agent}
\date{February 18, 2026}

\begin{document}

\maketitle

\section{Objective}
The goal of this experiment is to "explain" the price changes of Bitcoin (BTC) using a set of 16 diverse futures contracts (Indices, Energy, Metals, FX, Bonds, Agriculture) through an iterative decomposition process. This process identifies which global macro factors most strongly drive or correlate with 1-minute (resampled to 10-minute) BTC returns.

\section{Methodology}
We use \textbf{Orthogonal Matching Pursuit (OMP)}:
\begin{enumerate}
    \item Align 3 months of BTC 1-minute data with 16 futures.
    \item In each step, identify the future $F_i$ that has the highest correlation ($R^2$) with the current BTC residual.
    \item Perform linear regression $BTC_{res} = \beta F_i + \alpha$.
    \item Subtract the prediction from $BTC_{res}$ to update the residual.
    \item Orthogonalize all remaining futures with respect to the chosen future to ensure subsequent steps explain only new variance.
    \item Repeat until all futures are exhausted.
\end{enumerate}

\section{Results}
The following plot shows the marginal variance explained by each asset in sequential order.

\begin{center}
    \includegraphics[width=0.9\textwidth]{btc_decomposition_plot.png}
\end{center}

\subsection{Step-wise Contribution}
\begin{center}
\begin{tabular}{llr}
\toprule
Step & Future & Marginal $R^2$ \\
\midrule
1 & NQ=F (Nasdaq 100) & 0.0944 \\
2 & RTY=F (Russell 2000) & 0.0279 \\
3 & SI=F (Silver) & 0.0210 \\
4 & YM=F (Dow 30) & 0.0018 \\
5 & ES=F (S\&P 500) & 0.0013 \\
\bottomrule
\end{tabular}
\end{center}

\section{Analysis}
\begin{enumerate}
    \item \textbf{Tech Sensitivity:} The Nasdaq 100 (NQ=F) is the single largest "explainer" of BTC returns, accounting for nearly 10\% of the variance in 10-minute returns. This confirms the strong coupling between BTC and technology-heavy risk assets.
    \item \textbf{Small Caps and Metals:} Once the Nasdaq component is removed, the Russell 2000 (RTY=F) and Silver (SI=F) contribute an additional 2-3\% each. This indicates that BTC also behaves as a small-cap risk asset and a digital alternative to precious metals.
    \item \textbf{Diminishing Returns:} Beyond the top 3-5 factors, the marginal explanation of additional futures (Energy, Bonds, Ag) drops to near-zero ($<0.1\%$). This suggests that the "macro" component of BTC is highly concentrated in tech-equity and dollar-hedging factors.
    \item \textbf{Residual Variance:} Approximately 85\% of the variance in BTC 10-minute returns remains unexplained by these 16 futures, likely representing idiosyncratic crypto-market factors, liquidity flows, and news.
\end{enumerate}

\section{Conclusion}
The iterative linear decomposition successfully identified the primary global drivers of BTC price action. The "Nasdaq factor" dominates, but the significant residual variance confirms that BTC remains largely driven by its own internal market dynamics at the 10-minute frequency.

\end{document}
