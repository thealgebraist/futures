\documentclass{article}
\usepackage[utf8]{inputenc}
\usepackage{geometry}
\geometry{a4paper, margin=1in}
\usepackage{booktabs}
\usepackage{amsmath}

\title{Walk-Forward Profit Analysis: 8-Iteration Simulation}
\author{Gemini CLI Agent}
\date{February 18, 2026}

\begin{document}

\maketitle

\section{Methodology}
The analysis employed an 8-iteration walk-forward validation scheme. The test month was divided into 8 sequential folds. In each iteration, the Discontinuous PWL-FFNN ($N=128$) was retrained on all historical data and then simulated on the unseen fold. Starting capital was \$100.00.

\section{Results}
The simulation encountered immediate catastrophic failure in the first iteration.

\begin{center}
\begin{tabular}{lr}
\toprule
Metric & Value \\
\midrule
Starting Capital & \$100.00 \\
Iteration 0 PnL & -\$966.16 \\
Final Capital & \$0.00 \\
Status & \textbf{Account Liquidated} \\
\bottomrule
\end{tabular}
\end{center}

\section{Critical Analysis}
The rapid liquidation of the \$100 account highlights the extreme leverage and risk inherent in futures trading:
\begin{enumerate}
    \item \textbf{Margin vs. PnL:} A \$100 account is insufficient to withstand even a small correction in Micro ES futures. The simulation shows a loss of nearly \$1,000 in the first few days of the test set, which would have triggered immediate broker liquidation.
    \item \textbf{Model Bias:} The PWL-FFNN, while exhibiting low training error, likely failed to capture a sharp regime shift at the start of the walk-forward period, leading to incorrect directional bets.
    \item \textbf{Transaction Cost Erosion:} The cumulative impact of \$1.50 per trade on a high-frequency (10-min) signal is lethal for a sub-\$1,000 account.
\end{enumerate}

\section{Conclusion}
The 8-iteration walk-forward test confirms that while high-capacity models like the PWL-FFNN can fit historical data well, the combination of high-frequency noise and low starting capital makes futures trading statistically unsustainable for a \$100 account. A minimum capitalization of \$5,000+ is recommended to survive the volatility observed in the walk-forward period.

\end{document}
