\documentclass{article}
\usepackage[utf8]{inputenc}
\usepackage{amsmath}
\usepackage{graphicx}
\usepackage{booktabs}

\title{Monte Carlo FX Futures Estimator: Sum of Gaussian Walks}
\author{Antigravity Agent}
\date{February 18, 2026}

\begin{document}

\maketitle

\section{Stochastic Formulation}
The FX future price $S_t$ is modeled as a sum of two Gaussian walks (fast and slow) with piecewise drift and volatility, supplemented by cyclical seasonality and instantaneous Dirac shocks.

\subsection{Path Equation}
The discretized price update is defined as:
\begin{equation}
\Delta S_t = \sigma(t) \left( \sigma_{fast} \epsilon_{f,t} + \sigma_{slow} \epsilon_{s,t} \right) + \mu(t) \Delta t + A \omega \cos(\omega t + \phi) \Delta t + \text{Jump}_t
\end{equation}

Where:
\begin{itemize}
    \item $\mu(t)$ is the piecewise drift (regime-dependent).
    \item $\sigma(t)$ is the piecewise volatility scaling (second-order jump).
    \item $\epsilon \sim N(0, \sqrt{\Delta t})$ represents the Brownian increments.
    \item $\text{Jump}_t \sim \text{Poisson}(\lambda)$ represents discrete shocks.
\end{itemize}

\section{Simulation Results}
We performed $N=10,000$ simulations with $T=1.0$ and $\Delta t=0.01$.

\begin{table}[h]
\centering
\begin{tabular}{lr}
\toprule
Metric & Value \\
\midrule
Expected Terminal Value & 0.0489 \\
Standard Deviation & 0.0634 \\
95\% Confidence Interval & [0.0477, 0.0501] \\
\bottomrule
\end{tabular}
\caption{Monte Carlo Statistical Summary}
\end{table}

\begin{figure}[ht]
\centering
\includegraphics[width=0.8\textwidth]{mc_visualization.png}
\caption{Top: 10 representative paths showing regime switching at $T/2$. Bottom: Terminal value distribution.}
\end{figure}

\section{Conclusion}
The Sum of Walks framework effectively captures the multifractal nature of FX markets, showing how piecewise regime shifts in drift and volatility (second-order jumps) significantly impact the terminal value distribution.

\end{document}
